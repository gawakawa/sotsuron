\documentclass[uplatex]{suribt}
%\documentclass[oneside]{suribt}% 本文が * ページ以下のときに (掲示に注意)
\title{タイトル}
%\titlewidth{}% タイトル幅 (指定するときは単位つきで)
\author{長谷川 慧}
\eauthor{Hiraku Hasegawa}% Copyright 表示で使われる
\studentid{08-222021}
\supervisor{森畑明昌 准教授}% 1 つ引数をとる (役職まで含めて書く)
%\supervisor{指導教員名 役職 \and 指導教員名 役職}% 複数教員の場合,\and でつなげる
\handin{2024}{01}% 提出月. 2 つ (年, 月) 引数をとる
%\keywords{キーワード1, キーワード2} % 概要の下に表示される

\begin{document}
\maketitle%%%%%%%%%%%%%%%%%%% タイトル %%%%

\frontmatter% ここから前文
\begin{abstract}%%%%%%%%%%%%% 概要 %%%%%%%%
 ここに概要を書く.
\end{abstract}

\tableofcontents%%%%%%%%%%%%% 目次 %%%%%%%%

\mainmatter% ここから本文 %%% 本文 %%%%%%%%
\chapter{はじめに}
% 背景
% プログラム検証とは
プログラム検証とは、
プログラムの正しさを自動的に証明する試みのことである。
具体的には、プログラムとその仕様を入力として受け取り、
そのプログラムが仕様を満たしているかどうかを判定し、
仮に仕様を満たしていなければ反例とともに結果を出力する。
% プログラム検証の目的・意義


% 関係的検証とは
関係的検証とは、複数のプログラムの関係という形で表される仕様(関係的検証)
を検証するプログラム検証の一分野である。
% 普通の検証との共通点と相違点
通常の検証と比較すると、
プログラムと仕様を受け取ってプログラムが仕様を満たすかどうかを
判定する点は共通しているが、
対象とするプログラムが複数であり、
仕様がそれらの関係という形で表される点が異なる。% 言い回し変える
関係的検証においても不変条件を発見することが鍵となるが、% これより前に不変条件への言及が必要
一般に関係的検証問題を解くのに役立つ不変条件は複数のプログラムの変数に関する述語となるので、
探索空間が比較的大きくなる傾向があるところに難しさがある。% 文章の構造おかしい

% 既存手法 簡単に
関係的検証へのアプローチとしては、

% Unnoらの研究の概要 簡単に
Unnoらは、

% 目的
本研究では、PCSATによって関係的検証の問題がどの程度解けるかを調査する。
PCSATを用いた検証やPCSATの性能評価は
Unnoら以外には行われていない。
また、PCSAT用のベンチマークは、
9つの仕様に対してのみ行われている上に、
基本的な算術・論理演算から成る単純なプログラムに限られており
関数や配列を含むようなプログラムは含まれていない。
% 実際、ベンチマークの1つとして挙げられている配列への要素の挿入を意図したプログラム(arrayInsert)では、
% 配列を用いる(検証には不要な)記述を削除することで
% 配列を用いないプログラム
さらに、Unnoらの研究で示されたアプローチは複雑であるため、
PCSATの応用性を予測するのは困難である。

% PCSATを実際に使えるか確かめるため追試を行った -> あまり使い物にはならなかった
% ・benchmarkの追試 solverのupdateが原因なんじゃないか(想像)
% ・自作問題(3つ) ヒント
% ・array read-only
以上を踏まえて、PCSATの性能調査を行った。
まず、Unnoらの研究で用いられたベンチマークを使ってPCSATの追試を行った。
その結果、ベンチマークのうちいくつかの問題 % 数をはっきり言った方が良さそう
については解くことができないことがわかった。
これを受けて、解けなかった問題についてはベンチマークに自作のヒントを追加し、検証に成功した。
次に、自作の関係的検証問題を3つ作成し、それをPCSATで解くことができるかを調査した。
ここでも解くことができなかった問題があったので、% 言い回し変える
その問題についてはヒントを追加し、検証に成功した。
ここまでの検証問題はベンチマークにあるような単純な計算のみからなるものであったが、
PCSATの応用性を確かめるために配列から読み取った値を用いるプログラムの関係的検証を試みた。
これについては、配列を関数とみなした上で関数をinterpretedな述語変数に変換することで、
PCSATに解ける問題に帰着させ、ヒントを追加することで検証に成功した。

% 本論文の構成
本論文の構成は以下の通りである。
まず、2章で関係的検証の説明と既存手法について述べる。
次に、3章で本研究で行った実験の概要について述べる。
続いて4章において3章で行った実験の結果を考察し、
5章で本研究のまとめと今後の課題について述べる。

\chapter{先行研究}
\section{Relasional verifiation}
\section{Unnoら}

\chapter{方法・実験}
% 実験列挙
% \chapter{結果}
\chapter{考察}
\chapter{おわりに}
% \chapter{関連研究}

\backmatter% ここから後付


\bibliographystyle{abbrv}%           BibTeX を使う場合
\bibliography{references}% BibTeX を使う場合

% \appendix% ここから付録 %%%%% 付録 %%%%%%%
% \chapter{}
\end{document}