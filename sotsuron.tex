\documentclass[uplatex]{suribt}
%\documentclass[oneside]{suribt}% 本文が * ページ以下のときに (掲示に注意)
\title{タイトル}
%\titlewidth{}% タイトル幅 (指定するときは単位つきで)
\author{長谷川 慧}
\eauthor{Hiraku Hasegawa}% Copyright 表示で使われる
\studentid{08-222021}
\supervisor{森畑明昌 准教授}% 1 つ引数をとる (役職まで含めて書く)
%\supervisor{指導教員名 役職 \and 指導教員名 役職}% 複数教員の場合,\and でつなげる
\handin{2024}{01}% 提出月. 2 つ (年, 月) 引数をとる
%\keywords{キーワード1, キーワード2} % 概要の下に表示される

\begin{document}
\maketitle%%%%%%%%%%%%%%%%%%% タイトル %%%%

\frontmatter% ここから前文
\begin{abstract}%%%%%%%%%%%%% 概要 %%%%%%%%
 ここに概要を書く.
\end{abstract}

\tableofcontents%%%%%%%%%%%%% 目次 %%%%%%%%

\mainmatter% ここから本文 %%% 本文 %%%%%%%%
\chapter{はじめに}
\section{背景}
検証のこと
関係的検証とは
・普通の検証との共通点と相違点
海野ら 必要最小限のことを書く

\section{研究の目的}
pcsatの調査不足 
  benchmark少ない
  しかもprimitive
non-trivialな例はない(e.g. arrayInsert)
pcsatの複雑性によってスケーラビリティに想像つかない

実際問題使えるの?

使えないならヒントが必要なのでは?
・benchmarkの追試 solverのupdateが原因なんじゃないか(想像)
・自作問題(3つ) ヒント
・array read-only
\section{本論文の構成}
\section{}
\chapter{先行研究} % ここがデカくなりそう
\section{Unnoら}
\section{Relasionak verifiation}

\chapter{方法} % まとめる
\chapter{実験}
\section{結果}
\section{考察}

\chapter{おわりに}
% \chapter{関連研究}

\backmatter% ここから後付

\begin{thebibliography}{}%%%% 参考文献 %%%
 \bibitem{}
\end{thebibliography}
%\bibliographystyle{}%           BibTeX を使う場合
%\bibliography{.bib ファイル名}% BibTeX を使う場合

\appendix% ここから付録 %%%%% 付録 %%%%%%%
\chapter{}
\end{document}